\chapter{Possibles ampliacions al disseny}

\begin{description}
    \item \textbf{Utilitzaci� de diversos \textit{backend} alhora} \\
	Emulant el sistema que utilitza Linux per als diferents tipus de
	sistemes de fitxers, es podria incloure un sistema que
	permet�s registrar din�micament en forma de m�duls cadascun dels
	\textit{backend} sota demanda, de manera que en accedir a un cl�ster
	en el nostre arbre, se seleccionaria el \textit{backend} corresponent
	per a dur a terme la operaci� sol�licitada.
	
    \item \textbf{Recol�lecci� d'informaci� remota} \\
	Donades les limitacions d'\textit{OpenMosix} a l'hora d'accedir a la
	informaci� de nodes remots, es podrien implementar dos nous elements,
	un \textbf{recolector} d'informaci�, per a recollir la informaci� dels
	nodes remots, i un \textbf{informador}, que aniria a cadascun dels
	nodes remots als que volem poder accedir.

	Aix� doncs, la informaci� dels nodes remots seria la informaci� del
	node local d'all� on est� cadascun dels recolectors, salvant aix� el
	problema que ens oferia OpenMosix, i podent eliminar aleshores,
	tamb�, la limitaci� que nosaltres mateixos hem posat de crear
	processos des d'un sol node com a punt d'entrada.
\end{description}

