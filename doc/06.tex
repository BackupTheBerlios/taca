\chapter{Possibles ampliacions al disseny}

\begin{description}
    \item \textbf{Utilitzaci� de diversos \textit{backend} alhora} \\
	Emulant el sistema que utilitza Linux per als diferents tipus de
	sistemes de fitxers, es podria incloure un sistema que
	permet�s registrar din�micament en forma de m�duls cadascun dels
	\textit{backend} sota demanda, de manera que en accedir a un cl�ster
	en el nostre arbre, se seleccionaria el \textit{backend} corresponent
	per a dur a terme la operaci� sol�licitada.

    \item \textbf{Recol�lecci� d'informaci� remota} \\
	Donades les limitacions d'\textit{OpenMosix} a l'hora d'accedir a la
	informaci� de nodes remots, es podrien implementar dos nous elements,
	un \textbf{recolector} d'informaci�, per a recollir la informaci� dels
	nodes remots, i un \textbf{informador}, que aniria a cadascun dels
	nodes remots als que volem poder accedir.

	Aix� doncs, la informaci� dels nodes remots seria la informaci� del
	node local d'all� on est� cadascun dels recolectors, salvant aix� el
	problema que ens oferia OpenMosix, i podent eliminar aleshores,
	tamb�, la limitaci� que nosaltres mateixos hem posat de crear
	processos des d'un sol node com a punt d'entrada.

    \item \textbf{Arquitectura transl�cida} \\
	A vegades es pot donar el cas de qu� un sistema de clustering concret
	proporcioni uns serveis molt espec�fics d'aquest, de forma que des del
	\textit{frontend} es podria oferir una interf�cie gen�rica que permet�s al
	\textit{backend} definir aquesta part m�s espec�fica de l'arbre de directoris.
	
	Aquesta soluci�, per�, requeriria que, en major o menor mesura, el mateix
	\textit{backend} proporcion�s les seves pr�pies funcions de generaci� de
	l'arbre espec�fic, proporcionant quelcom semblant a una versi� redu�da del
	\textit{frontend}.
	
	Duent-ho m�s enll�, per�, es podria arribar a crear una s�rie d'estructures i
	funcions gen�riques que evitessin aquesta feina a la capa d'abaix, tot i que
	un sistema de clustering sol anar molt lligat al sistema operatiu i, comparant
	les dues APIs de sistemes de fitxers que hem mirat (el VFS de Linux i el de
	GNU/Hurd), hem vist que, tret de petits detalls, les funcionalitats que
	ofereixen s�n les mateixes o molt similars, de manera que no seria molt
	dif�cil generalitzarn-ne una interf�cie i amagar-ne els detalls en camps
	dedicats a la la informaci� espec�fica de cada sistema (tal com ja fa Linux
	amb els sistemes de fitxers, que reserva un camp que �s un simple apuntador a
	dins dels inodes per a que cada sistema de fitxers hi posi la informaci� que
	consideri necess�ria.
\end{description}
