\begin{document}

\chapter{Software de clustering existent}

\section{Necessitats generals}
(qu� ens ha d'oferir la capa inferior) \\
(poder migrar processos, poder veure on s�n, ...) \\
(qu� nosaltres VOLEM poder fer)

\subsection {A nivell de cluster...}
\begin{itemize}
\item Consultar quantitat de nodes del cluster
\item Consultar carrega total del cluster
\end{itemize}

\subsection {A nivell de node...}
\begin{itemize}
\item Quantitat de CPU's d'un node
\item Consultar carrega del node
\item Consultar estat de memoria
\item Bloquejar l'arribada de nous processos remots
\item Bloquejar la migracio de processos locals
\item Activar/Desactivar la migraci� automatica de processos
\item Demanar que els processos que ha creat un node i han migrat, tornin  a
aquest node (come back home)
\item Expulsar els "processos convidats" a casa
\item Saber quin identificador te un node
\item Modificar/Consultar la velocitat d'un node
\item Consultar informacio de processos (resum de l'estat de processos locals, i
remots)
\item Consultar resum de l'estat del node
\end{itemize}

\subsection {A nivell de proces...}
\begin{itemize}
\item Moure processos d'un node a un altre
\item Bloquejar un proces en un node concret (/proc.....lock)
\item Crear un nou proces en una maquina (crear un fitxer)
\item Matar un proces (esborrar un fitxer)
\item Copiar un proces??
\item Modificar/veure memoria del proces (/proc....mem)
\item Consultar la linia de comandes amb la que s'ha ex/proc....cmdline)
\item Consultar estat del proces (/proc....status)
\item Consultar quants cops s'ha migrat el proces
\item Consultar on s'esta computant actualment
\item Consultar quin node el va crear
\item Saber l'identificador de proces (ID global)
\item Saber l'identificador local d'un proces (ex: PID)
\item Consultar resum de us de CPU
\item Consultar resum de us de Memoria
\end{itemize}


\section{Qu� ens ofereixen els sistemes}
(qu� proporciona a una capa superior) \\
(si ja proporciona una abstracci� semblant a la que volem...) \\
(qu� nosaltres PODEM fer)
                
\subsection{Linux}

\subsubsection{OpenMosix}

\paragraph{A nivell de cluster...}
\begin{itemize}
\item Consultar quantitat de nodes del cluster
	Consultant el numero de subdirectoris de tipus NODE del directori cluster.
\item Consultar carrega total del cluster
	Mitjana ponderada de la carrega dels nodes que conte el cluster.
\end{itemize}

\paragraph{A nivell de node...}
\begin{itemize}
\item Quantitat de CPU's d'un node
	Consultant el fitxer /proc/hpc/nodes/\textless opMosix\_ID\textgreater/cpus
\item Consultar carrega del node
	Consultant el fitxer /proc/hpc/nodes/\textless opMosix\_ID\textgreater/load
\item Consultar estat de memoria
	Consultant el fitxer /proc/hpc/nodes/\textless opMosix\_ID\textgreater/mem
\item Bloquejar l'arribada de nous processos remots (per altres nodes que no
sigui el local??muntar /proc..admin dels altres??)
	Consultant el fitxer /proc/hpc/admin/block
\item Bloquejar la migracio de processos locals
	Bloquejar cada proc�s del node (mirar seg�ent secci�)
\item Activar/Desactivar la migraci� automatica de processos
	Modificar el fitxer /proc/hpc/admin/stay
\item Demanar que els processos que ha creat un node i han migrat, tornin  a
aquest node (come back home)
	Modificar el fitxer /proc/hpc/admin/bring (com???)
\item Expulsar els "processos convidats" a casa
	Modificar el fitxer /proc/hpc/admin/expel
\item Saber quin identificador te un node
	Consultar el fitxer /proc/hpc/admin/mospe (o mirar el nom del directori del
node)
\item Modificar/Consultar la velocitat d'un node
	Consultar el fitxer /proc/hpc/admin/speed
\item Consultar informacio de processos (resum de l'estat de processos locals, i
remots)
	Generar un resum consultant la informaci� de cada proc�s del node amb les
operacions
	de consulta de informaci� de proces (mirar seg�ent secci�)
\item Consultar resum de l'estat del node
	Generar un resum amb la informaci� sobre el node descrita a les funcionalitats
anteriors.
\end{itemize}

\paragraph{A nivell de proces...}
\begin{itemize}
\item Moure processos d'un node a un altre
	Fer un migrate del proces que volem moure al node on ha d'anar
	=> Esborrar els fitxers relatius al proces del node origen
	=> Generar els fitxers relatius al proces al node desti
\item Bloquejar un proces en un node concret (/proc.....lock)
	Modificar el fitxer /proc/[PID]/lock
\item Crear un nou proces en una maquina (crear un fitxer)
	Posar en execucio el proces al node on volem que s'executi
	=> Generar els fitxers relatius al proces en aquest node.
\item Matar un proces (esborrar un fitxer)
	Fer un "kill" del proces
	=> Esborrar els fitxers relatius al proces al node on s'esta executant.
\item Copiar un proces??
\item Modificar/veure memoria del proces (/proc....mem) (...?)
	Consultar el fitxer /proc/[PID]/mem
\item Consultar la linia de comandes amb la que s'ha ex/proc....cmdline) (...?)
	Consultar el fitxer /proc/[PID]/cmdline
\item Consultar estat del proces (/proc....status) (...?)
	Consultar el fitxer /proc/[PID]/status
\item Consultar quants cops s'ha migrat el proces (...?)
	Consultar el fitxer /proc/[PID]/nmigs
\item Consultar on s'esta computant actualment
	Consultar el fitxer /proc/[PID]/where (o mirar el directori pare)
\item Consultar quin node el va crear
	Consultar el fitxer /proc/hpc/remote/from (de quin???)
\item Saber l'identificador de proces (ID global) (el generem nosaltres?? cada
cop que creem un proces??)
\item Saber l'identificador local d'un proces (ex: PID) (...?)
	Consultar el directori que representa el proces a /proc/[PID]
\item Consultar resum de us de CPU (...?)
	Per locals -> top -p PID
	Per remots -> consultar fitxer /proc/hpc/remote/stats
\item Consultar resum de us de Memoria (...?)
	idem que l'anterior
\end{itemize}

\textbf{\textit{\underline{...? = volem mirar aquesta informacio de processos
que son a altres maquines?}}} \\

L'escenari sera oferir un VFS nomes al node d'entrada, i els migrate... de
OPMosix
no treuen la informacio dels processos de /proc...


\subsubsection{Beowulf}
\nocite{Beowulf-MiniHowto}
Tal com s'ha explicat anteriorment, Beowulf �s una arquitectura i no un paquet
de software, de manera que hi ha diverses eines o distribucions que permeten
contru�r aquesta arquitectura:

\begin{itemize}
\item Cluster Management Utility (CMU) de Compaq
\item Projecte OSCAR (Open Source Cluster Application Resources)
\item ROCKS Clustering Toolkit.
\item Patagonia Cluster Project
\item SMILE Cluster Management System (SCMS) i les seves eines per a Beowulf
\item Projecte SCore Cluster System Software (Score)
\item Software ALINKA Linux Clustering
\end{itemize} 

Degut a la gran varietat d'eines d'administraci� i de sistemes de comunicaci�
que utilitzen els programadors (PVM, MPI, etc.) en aquesta arquitectura
(precisament per ser una arquitectura i no un software concret), no ens ha estat
possible definir les caracter�stiques concretes de Beowulf i, per tant, en la
resta del treball ens basarem en les experi�ncies concretes d'OpenMosix i la
teoria sobre el que ofereix un sistema de computaci� distribu�da.


\subsection{Microkernels (Mach/L4)}
Tal com s'ha explicat a l'apartat 2, els cl�sters existents sobre Microkernels
segueixen la mateixa arquitectura que BeoWulf, no existeix cap tipus d'est�ndard
ni d'API que ens serveixi per definir les caracter�stiques concretes de cl�sters
en Microkernels, ja que totes les implementacions que hem trobat s�n a nivell
d'investigaci� i/o recerca.



\subsection{Altres Sistemes}
����No s'ha trobat res????

Comentar-ho???????????????



\end{document}
