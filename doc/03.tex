\chapter{Software de clustering existent}

\section{Necessitats generals}
(qu� ens ha d'oferir la capa inferior)
(poder migrar processos, poder veure on s�n, ...)
(qu� nosaltres VOLEM poder fer)

\section{Qu� ens ofereixen els sistemes}
(qu� proporciona a una capa superior)
(si ja proporciona una abstracci� semblant a la que volem...)
(qu� nosaltres PODEM fer)
                
\subsection{Linux}

\subsubsection{OpenMosix}

\subsubsection{Beowulf}
\nocite{Beowulf-MiniHowto}
Tal com s'ha explicat anteriorment, Beowulf �s una arquitectura i no un paquet
de software, de manera que hi ha diverses eines o distribucions que permeten
contru�r aquesta arquitectura:

\begin{itemize}
\item Cluster Management Utility (CMU) de Compaq
\item Projecte OSCAR (Open Source Cluster Application Resources)
\item ROCKS Clustering Toolkit.
\item Patagonia Cluster Project
\item SMILE Cluster Management System (SCMS) i les seves eines per a Beowulf
\item Projecte SCore Cluster System Software (Score)
\item Software ALINKA Linux Clustering
\end{itemize} 

Degut a la gran varietat d'eines d'administraci� i de sistemes de comunicaci�
que utilitzen els programadors (PVM, MPI, etc.) en aquesta arquitectura
(precisament per ser una arquitectura i no un software concret), no ens ha estat
possible definir les caracter�stiques concretes de Beowulf i, per tant, en la
resta del treball ens basarem en les experi�ncies concretes d'OpenMosix i la
teoria sobre el que ofereix un sistema de computaci� distribu�da.

\subsection{Microkernels (Mach/L4)}

\subsection{...}
