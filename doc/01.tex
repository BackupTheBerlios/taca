\chapter{Qu� i perqu� ho volem fer?}

El que pretenem fer �s una interf�cie per al control i administraci� d'un
sistema de clustering a trav�s del sistema de fitxers.

Per exemple, per migrar un proc�s d'un node a un altre, seria tan f�cil com
moure un fitxer d'un directori a un altre (representant cada directori un node
diferent i cada proc�s representat per un fitxer) amb el navegador de fitxers
preferit.

D'aquesta manera l'usuari no ha de tenir coneixements sobre com funcionen les
eines de clustering que li ofereix el sistema on treballa, de forma que
l'abstracci� que proposem li permeti treballar amb diferents sistemes des d'una
interf�cie comuna.

Hem anomenat al sistema TACA, que ve de l'angl�s Transparent Architecture for
Cluster Administration (Arquitectura Transparent per a l'Administraci� de
Cl�sters).

Tant la documentaci� com la implementaci� es troben disponibles a la p�gina del
projecte, que es troba a \hyperlink{http://developer.berlios.de/projects/taca}
{http://developer.berlios.de/projects/taca}.

